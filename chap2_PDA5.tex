\input{header}

\AtBeginSubsection[]
{
	\begin{frame}<beamer>
		\frametitle{Outline}
		\tableofcontents[current,currentsubsection]
	\end{frame}
}

\begin{document}

\begin{frame}[allowframebreaks]
  \frametitle{The Overall Procedure}
  \begin{itemize}
  \item Given
    \begin{equation*}
    P=(Q,\Sigma, \Gamma, \delta, q_0, \{q_{accept}\})
  \end{equation*}
  \item Construct a CFG $G$
    \begin{equation*}
\text{var}(G) =\{A_{pq}\mid p, q \in Q\}
\end{equation*}
\item Start variable:
  \begin{equation*}
  A_{q_0, q_{accept}}
\end{equation*}

\item Rules: see earlier slides
  \end{itemize}
\end{frame}

\begin{frame}[allowframebreaks]
  \frametitle{Needed modifications of PDA}
  \begin{itemize}
  \item Recall we need PDA to satisfy
    \begin{enumerate}
    \item Single accept state
    \item Stack empty before accepting
    \item Each transition push or pop, but not both
    \end{enumerate}
  \item Let's handle the first two together:
    single accept and stack empty before accepting:

  \item A new start $q_s \rightarrow q_{s'}$ with $\epsilon, \epsilon
\rightarrow \$ $

\item For any $q \in F$, we have
  $\epsilon, a\rightarrow \epsilon $ back to $q$, $\forall a$.
  This pops things out before accepting a string
\item Then from any $q \in F$, we do $\epsilon, \$ \rightarrow
\epsilon$ to $q_a$.
\item $q \in F$ are no longer accept states
\item See the illustration in the following figures
\item Original PDA:

  \begin{tikzpicture}
    \node[state,initial] (q_0) {$q_{s'}$};
\node[state,draw=none,fill=white] (q_d) [right of=q_0, xshift=-1cm] {$\cdots$};    
\node[state] (q_1) [above right of=q_d, accepting, yshift=-1cm] {$q_1$};
\node[state] (q_2) [below right of=q_d, accepting, yshift=1cm] {$q_2$};
  \end{tikzpicture}

  New:
  \begin{tikzpicture}
    \node[state] (q_0) {$q_{s'}$};
    \node[state,initial above, left of=q_0] (q_s) {$q_{s}$};    
\node[state,draw=none,fill=white] (q_d) [right of=q_0, xshift=-1.3cm] {$\cdots$};    
\node[state] (q_1) [above right of=q_d, yshift=-1.2cm,xshift=-0.3cm] {$q_1$};
\node[state] (q_2) [below right of=q_d, yshift=1.2cm,xshift=-0.3cm] {$q_2$};
\node[state,accepting, right of=q_d, xshift=2cm] (q_a) {$q_{a}$};

  \path 
  (q_s) edge[above]  node {$\epsilon, \epsilon \rightarrow \$ $} (q_0)
  (q_1) edge[loop above]  node {$\epsilon, a \rightarrow \epsilon$\quad $\epsilon, b \rightarrow \epsilon$ } (q_1)
  (q_2) edge[loop below]  node {$\epsilon, a \rightarrow \epsilon$\quad $\epsilon, b \rightarrow \epsilon$} (q_2)  
  (q_1) edge[above]  node {$\epsilon, \$ \rightarrow \epsilon$} (q_a)
  (q_2) edge[below]  node {$\epsilon, \$ \rightarrow \epsilon$} (q_a);
  \end{tikzpicture}
\item  To have each transition push or pop, but not both,
change
\begin{center}
$q_1 \rightarrow q_2$ with $a, a \rightarrow b$ 
\end{center}
to
\begin{center}
  \begin{tabular}{l}
$q_1 \rightarrow q_3, a, a \rightarrow \epsilon$\\
$q_3 \rightarrow q_2, \epsilon, \epsilon \rightarrow b$  
  \end{tabular}
\end{center}
and change 
\begin{center}
$q_1\rightarrow q_2, a, \epsilon \rightarrow \epsilon$ 
\end{center}
to
\begin{center}
  \begin{tabular}{l}
$q_1 \rightarrow q_3, a, \epsilon \rightarrow ?$ \\
$q_3 \rightarrow q_2, \epsilon, ? \rightarrow \epsilon$
  \end{tabular}
\end{center}
  \end{itemize}
\end{frame}

% \begin{frame}[allowframebreaks]
%   \frametitle{Claim 2.30, ``$\Rightarrow$'' of (\ref{eq:emptystack})}
%   \begin{itemize}  
% \item $A_{pq}$ generates $x \Rightarrow p \rightarrow q$ with
% empty stack
% \item formal proof by induction
% on \# steps to derive $x$

% basis: 1-step derivation
% \begin{equation*}
%   A_{pp} \rightarrow \epsilon
% \end{equation*}

% is our only such rule

% trivial: $p$ with empty stack to $p$ with empty stack

% induction: ok rules with derivation length
% $\leq k$

% $A\stackrel{*}{\Rightarrow} x, k+1$ steps

% ($\stackrel{*}{\Rightarrow}$ defined on p 94)
% \item first step
%   \begin{equation*}
%     A_{pq}\Rightarrow a A_{rs}b
% \mbox{ or } A_{pq}
% \Rightarrow A_{pr} A_{rq}
%   \end{equation*}
%   \begin{enumerate}
%   \item $x=ayb,p\rightarrow r, $ stack $\{\$\}\rightarrow\{a\}$

% induction $r\rightarrow s$ with $y$ and stack $\{a\}$

% $s\rightarrow q$, stack $\{a\}\rightarrow \{\$\}$
% \item $x=yz$

% $A_{pr}\stackrel{*}{\Rightarrow}y,
% A_{rq}\stackrel{*}{\Rightarrow}z\leq k$ steps

%   \end{enumerate}
% \end{itemize}\end{frame} \begin{frame}[allowframebreaks] \frametitle{Claim 2.31: ``$\Leftarrow$'' of (\ref{eq:emptystack})}
%   \begin{itemize}
% \item ($p$ empty stack)
% $\stackrel{x}{\rightarrow}$
% ($q$ empty stack)
% $\Rightarrow A_{pq}$ generates $x$
% \item induction on \#steps by PDA

% basic: 0 step

% $p\rightarrow p$ with $\epsilon$

% $A_{pp}\rightarrow \epsilon$ in G, OK

% induction: $k$ ok $k+1$ ?

% $p\rightarrow q$ with $x$ , empty stack
% \begin{enumerate}
% \item 
% empty, only beginning and end
% \item empty in the middle as well
% \end{enumerate}
% \begin{enumerate}
% \item symbol pushed at $p=$
% popped at $q$

% (not empty in the middle)

% call it $t$

% $p$: input $a$ , $q$: input $b$

% $(r,t)\in \delta(p,a,\epsilon), 
% (q,\epsilon)\in \delta(s,b,t)$

% $\Rightarrow \rightarrow a A_{rs} b$ in G

% $x=ayb$

% by induction $A_{rs}\stackrel{*}{\Rightarrow}y$ so

% $A_{pq}\stackrel{*}{\Rightarrow}x$


% \item 
% Let $r$ a state stack empty

% $x=yz$

% $p\rightarrow r,r\rightarrow q \leq k$ steps

% by induction $A_{pr}\stackrel{*}{\Rightarrow}y,
% A_{rq}\stackrel{*}{\Rightarrow} z$

% our construction $A_{pq}\rightarrow A_{pr}A_{rq}$ in G

% $A_{pq}\stackrel{*}{\Rightarrow} x$
% \end{enumerate}
% \end{itemize}\end{frame}

\begin{frame}[allowframebreaks] \frametitle{Regular language is context Free}
  \begin{itemize}
\item We roughly know this but didn't give a formal proof. Here are the steps
\item Regular language $\Rightarrow$ recognized by DFA (in Chapter 1)
\item DFA is a PDA
\item Thus regular language recognized by PDA
\item Then any regular language is context free (by the proof in this chapter)

\end{itemize}\end{frame} \begin{frame}[allowframebreaks] \frametitle{Non-context free languages}
  \begin{itemize}
\item There are such languages
\item We omit the discussion

\end{itemize}\end{frame}

\end{document}
