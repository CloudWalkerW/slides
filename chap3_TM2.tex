\input{header}

\AtBeginSubsection[]
{
	\begin{frame}<beamer>
		\frametitle{Outline}
		\tableofcontents[current,currentsubsection]
	\end{frame}
}

\begin{document}

\begin{frame}[allowframebreaks] \frametitle{Example 3.7}
  \begin{itemize}
\item Consider the following language
  \begin{equation*}
\{0^{2^n}\mid n \geq 0\}
\end{equation*}
Strings in this language are
\begin{equation*}
0,00,0000,00000000, \ldots
\end{equation*}
\item Idea: crossing off every other 0 and the remaining string should
  still have even length
\item Example:
  \begin{eqnarray*}
&& 0\underline{0} 0\underline{0}\\
&& 0\underline{0}\\
&& 0
  \end{eqnarray*}
\item Procedure
  \begin{enumerate}
  \item left $\rightarrow$ right, mark every other 0
  \item if in step 1, only one 0 left, then accept
  \item if in step 1, odd \# 0 left, then reject
\item move head to the beginning
\item go back to stage 1
  \end{enumerate}

\item Formal definition

\item [] $Q=\{q_1, q_2, q_3, q_4, q_5, q_{accept}, q_{reject}\}$

\item [] $\Sigma=
  \{0\}$

\item[] $\Gamma=\{0, x, \sqcup\}$
\item The diagram
\end{itemize}
\begin{tikzpicture}
\node[state,initial above] (q_1) {$q_1$};
\node[state] (q_2) [right of=q_1,xshift=0.5cm] {$q_2$};
\node[state] (q_3) [right of=q_2,xshift=1.7cm] {$q_3$};
\node[state] (q_4) [below of=q_3, yshift=0cm, xshift=0cm] {$q_4$};
\node[state] (q_5) [above right of=q_2, yshift=0cm] {$q_5$};
\node[state] (q_r) [below of=q_1, yshift=0cm] {$q_r$};
\node[state] (q_a) [below of=q_2, yshift=0.7cm] {$q_a$};
 \path 
 (q_1) edge[below]  node {$0 \rightarrow \sqcup$, R} (q_2)
 (q_1) edge[right]  node {
   \begin{tabular}{l}
     $\sqcup \rightarrow$ R\\
     x   $ \rightarrow$ R\\
   \end{tabular}} (q_r)
 (q_2) edge[loop above]  node {x $ \rightarrow $ R} (q_2)
 (q_2) edge[below]  node {$0 \rightarrow $ x, R} (q_3)
 (q_2) edge[right]  node {$\sqcup \rightarrow $ R} (q_a) 
 (q_3) edge[left, bend right=15]  node {$0 \rightarrow $ R} (q_4)
 (q_3) edge[loop right]  node {x $ \rightarrow $ R} (q_3)
 (q_3) edge[right]  node {$\sqcup \rightarrow $ L} (q_5) 
 (q_4) edge[right, bend right=15]  node {$0 \rightarrow $ x, R} (q_3)
 (q_4) edge[loop left]  node {x $ \rightarrow $ R} (q_4)
 (q_4) edge[above, bend left=20]  node {$\sqcup \rightarrow $ R} (q_r)
 (q_5) edge[loop right]  node {
   \begin{tabular}{l}
     $0 \rightarrow $ L\\
     x $\rightarrow$ L
   \end{tabular}
 } (q_5)
 (q_5) edge[right]  node {$\sqcup \rightarrow$ R } (q_2)    
;
  \end{tikzpicture}
\begin{itemize}

\item $0 \rightarrow R\equiv 0 \rightarrow 0, R$

\item Consider the input 0000
  \begin{center}
  \begin{tabular}{lllll}
    $q_1 0000$ & $\sqcup q_2 000$ & $\sqcup x q_3 00$ & $\sqcup x 0 q_4 0$
    & $\sqcup x 0 x q_3$                                         \\
    $\sqcup x 0 q_5 x$ & $\sqcup x q_5 0 x$ & $\sqcup q_5 x 0 x$
    &  $q_5 \sqcup x 0 x$ & $\sqcup q_2  x 0 x$ \\
$\sqcup x q_2  0 x$ & $\sqcup x x q_3  x$ & $\sqcup x x x q_3  \sqcup $
& $\sqcup x x q_5 x $ & $\sqcup x q_5 x x $\\
$\sqcup q_5 x x x $ & $q_5 \sqcup  x x x $ & $\sqcup  q_2 x x x $
& $\sqcup  x q_2 x x $ & $\sqcup  x x q_2 x $ \\
$\sqcup  x x x q_2$ & $\sqcup  x x x \sqcup q_a$ &&&
  \end{tabular}
\end{center}
\item  The $\delta$ function:

  \begin{center}
\begin{tabular}{c|ccc}
& 0 & x & $\sqcup$\\ \hline
$q_1 $ & $q_2,\sqcup,R$ & $q_{reject},x,R$ & $q_{reject},\sqcup,
R$\\
  $q_2$ & $q_3,x,R$ & $q_2,x,R$ & $q_{accept},\sqcup,R$
  \\
  $\vdots$ &&&
\end{tabular}
\end{center}

\item No need to have rows for $q_{accept},q_{reject}$

\item [] $\Rightarrow$ accepting/rejecting takes immediate effect

\item Now a deterministic TM

\item [] We can have nondeterministic TM later

\item [] They are equivalent

\item Main idea of $\delta$:

\item $q_1: $ mark the start by
$\sqcup$
\begin{itemize}
\item first element must be 0, otherwise, reject
\item Using $\sqcup$, so the start is known
\end{itemize}
\item $q_2 \rightarrow q_3$: handle initial 00

\item $q_3\rightarrow q_4 \rightarrow q_3$: sequentially $00 \rightarrow 0x$

  \begin{itemize}
  \item If not pairs (e.g., 0x0x0x), fails
  \item This is the place of checking if \# of remained zeros is even
  \end{itemize}
\item $q_3 \rightarrow q_5 \rightarrow q_2 $ back to beginning

% \item Why $q_2, q_3$ cannot be combined

% \item [] $q_3$ sees $\sqcup$: moves left

% \item [] $q_2$ sees $\sqcup$: first $\sqcup$ stop

\item First 0 (or $\sqcup$) is considered the single
final 0 
\begin{equation*}
q_2 \rightarrow \cdots
\rightarrow q_2 \rightarrow \cdots \rightarrow
q_{accept}
\end{equation*}
check if a single 0 is left in the string
% \item Exercise 

% 000000

% Even fewer steps than 0000

\end{itemize}\end{frame}

\end{document}

%%% Local Variables:
%%% mode: latex
%%% TeX-master: t
%%% End:

