\input{header}

\AtBeginSection[]
{
	\begin{frame}<beamer>
		\frametitle{Outline}
		\tableofcontents[current,currentsubsection]
	\end{frame}
}

\begin{document}

\begin{frame}[allowframebreaks] \frametitle{Example:
Fig 1.57    $(ab\cup a)^*$}

\begin{tabular}{rc}
  $a$ &
  \begin{tikzpicture}[scale=0.7, every node/.style={scale=0.7}]
\node[state, initial] (q1) {};
\node[state, accepting, right of=q1] (q2) {};

\draw (q1) edge[above] node{$a$} (q2);
\end{tikzpicture}
  \\
  $b$ &
  \begin{tikzpicture}[scale=0.7, every node/.style={scale=0.7}]
\node[state, initial] (q1) {};
\node[state, accepting, right of=q1] (q2) {};

\draw (q1) edge[above] node{$b$} (q2);
\end{tikzpicture}
  \\
  $ab$
      &
  \begin{tikzpicture}[scale=0.7, every node/.style={scale=0.7}]
    \node[state, initial] (q1) {};
    \node[state, right of=q1] (q2) {};
\node[state, right of=q2] (q3) {};        
\node[state, accepting, right of=q3] (q4) {};

\draw (q1) edge[above] node{$a$} (q2);
\draw (q2) edge[above] node{$\epsilon$} (q3);
\draw (q3) edge[above] node{$b$} (q4);
\end{tikzpicture}
  \\
  $ab \cup a$
      &
\begin{tikzpicture}[scale=0.7, every node/.style={scale=0.7}]
    \node[state, initial] (q0) {};          
    \node[state, above right of =q0, yshift=-1.4cm] (q1) {};
    \node[state, right of=q1] (q2) {};
\node[state, right of=q2] (q3) {};        
\node[state, accepting, right of=q3] (q4) {};
\node[state, below right of =q0, yshift=1.4cm] (q5) {};
\node[state, accepting, right of=q5] (q6) {};

\draw (q1) edge[above] node{$a$} (q2);
\draw (q2) edge[above] node{$\epsilon$} (q3);
\draw (q3) edge[above] node{$b$} (q4);

\draw (q5) edge[above] node{$a$} (q6);
\draw (q0) edge[above] node{$\epsilon$} (q1);
\draw (q0) edge[below] node{$\epsilon$} (q5);
\end{tikzpicture}
\end{tabular}


 $(ab \cup a)^*$
        \begin{tikzpicture}[scale=0.7, every node/.style={scale=0.7}]
    \node[state, initial, accepting] (q00) {};                    
    \node[state, right of=q00] (q0) {};          
    \node[state, above right of =q0, yshift=-1.45cm] (q1) {};
    \node[state, right of=q1] (q2) {};
\node[state, right of=q2] (q3) {};        
\node[state, accepting, right of=q3] (q4) {};
\node[state, below right of =q0, yshift=1.45cm] (q5) {};
\node[state, accepting, right of=q5] (q6) {};

\draw (q00) edge[above] node{$\epsilon$} (q0);
\draw (q1) edge[above] node{$a$} (q2);
\draw (q2) edge[above] node{$\epsilon$} (q3);
\draw (q3) edge[above] node{$b$} (q4);

\draw (q5) edge[above] node{$a$} (q6);
\draw (q0) edge[above] node{$\epsilon$} (q1);
\draw (q0) edge[below] node{$\epsilon$} (q5);
\draw (q4) edge[bend right=35, above] node{$\epsilon$} (q0);
\draw (q6) edge[bend left=45, below] node{$\epsilon$} (q0);
\end{tikzpicture}

\begin{itemize}
\item Such a construction usually does not give an
  NFA with the smallest number of
states
\item This example $\Rightarrow$ can be by 2 states

  \begin{center}
\begin{tikzpicture}
    \node[state, initial, accepting] (q1) {$q_1$};                    
    \node[state, right of=q1] (q2) {$q_2$};          
    \draw (q1) edge[loop above] node{$a$} (q1);
    \draw (q1) edge[above, bend left] node{$a$} (q2);
    \draw (q2) edge[below, bend left] node{$b$} (q1);        
\end{tikzpicture}
\end{center}
  \begin{equation*}
    (\{q_1, q_2\}, \{a,b\}, \delta, q_1, \{q_1\})
  \end{equation*}
  \begin{center}
  \begin{tabular}{c|ccc}
& a & b & $\epsilon$\\ \hline
$q_1$ & $\{q_1,q_2\}$ & $\emptyset$ & $\emptyset$\\
$q_2$ & $\emptyset$ & $\{q_1\}$ &$\emptyset$
  \end{tabular}
\end{center}
\end{itemize}\end{frame}

\begin{frame}[allowframebreaks] \frametitle{Regular $\Rightarrow$  a regular
    expression: a simple example}
  \begin{itemize} 
  \item Given an NFA, how can we convert it to a
    regular expression?
\item Example:

  \begin{center}
    \begin{tikzpicture}
    \node[state, initial] (1) {1};                    
    \node[state, accepting, right of=1] (2) {2};          

    \draw (1) edge[above] node{$b$} (2);
\draw (1) edge[loop above, left] node{$a$} (1);    
\draw (2) edge[loop above, right] node{$a,b$} (2);    
\end{tikzpicture}
  \end{center}
  
\item Quickly we see that this corresponds to
  \begin{equation*}
  a^*b(a\cup b)^*
\end{equation*}
\item However, in other situations we may not
  easily see what the regular
  expression is
\end{itemize}\end{frame}

\end{document}
